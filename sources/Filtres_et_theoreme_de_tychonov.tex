\documentclass{article}
\usepackage[utf8]{inputenc}
\usepackage{prelude}

\title{Filtres et théorème de Tychonov}
\author{Shika}
\date{Juin 2022}
\begin{document}

\maketitle


\section{Filtres et ultrafiltres}
\begin{definition}
    Un filtre sur $X$ est une partie $\mathcal{F} \subset \mathcal{P}(X)$ qui vérifie:
    \begin{enumerate}
        \item $\varnothing \not\in \mathcal{F}, X \in \mathcal{F}$
        \item $A, B \in \mathcal{F} \implies A \cap B \in \mathcal{F}$
        \item $A \in \mathcal{F} \land A \subset B \implies B \in \mathcal{F}$
    \end{enumerate}
\end{definition}

L'intuition derrière cette définition est qu'un filtre défini une notion de grosseur.\\
Ainsi, la première condition stipule que l'ensemble vide n'est pas gros et que l'ensemble entier l'est, la seconde que l'intersection de deux grosses parties de $X$ est encore une grosse partie de $X$ et la dernière qu'un sur-ensemble d'une grosse partie est aussi gros.\\

Un premier exemple de filtre sur un ensemble $X$ est le filtre trivial, donnée par $\mathcal{F} = \{X\}$.\\ Un exemple moins trivial est donné par le filtre principal associé à une partie non-vide $A \subset X$, donné par $\mathcal{F}_A = \{B \subset X \mid A \subset B\}$.\\
On peut aussi considérer le filtre cofini sur un ensemble $X$ infini, donné par $\mathcal{F}_{\text{cofinite}} = \{A \subset X \mid X\setminus A \text{ est fini}\}$.\\
Un dernier exemple, ou plutôt une construction, est l'intersection d'une famille quelconque $(\mathcal{F}_i)_{i \in I}$ de filtres, qui est encore un filtre (exercice).\\
L'union de deux filtres n'est cependant pas toujours un filtre.
Pour pouvoir considérer le filtre engendré par un ensemble de parties $\mathcal{B} \subset \mathcal{P}(X)$, il est clair qu'on a besoin d'avoir
$$\forall B_1, \ldots, B_n \in \mathcal{B}, \bigcap_{i=1}^{n}{B_i} \neq \varnothing$$
On dit que $\mathcal{B}$ a la propriété de l'intersection finie. Cette condition est aussi suffisante, puisque $\langle B \rangle = \{A \subset X \mid \exists B \in \mathcal{B}, B \subset A\}$ est alors un filtre, le plus petit contenant $\mathcal{B}$.

\begin{definition}
    Si $\mathcal{B} \subset \mathcal{P}(X)$ a la propriété de l'intersection finie, alors le filtre engendré par $B$ est donné par $$\langle B \rangle = \{A \subset X \mid \exists B \in \mathcal{B}, B \subset A\}$$
\end{definition}

\begin{definition}
    Un filtre $\mathcal{F}_1$ est plus fin qu'un filtre $\mathcal{F}_2$ si $\mathcal{F}_2 \subset \mathcal{F}_1$
\end{definition}

On peut se demander ce à quoi ressemble un filtre "saturé", autrement dit maximal pour l'inclusion. On montre que c'est en fait équivalent à la définition suivante, qui, selon l'analogie vue plus haut, ajoute la condition qu'une partie est soit grosse, soit de complémentaire gros:

\begin{definition}
    Un ultrafiltre sur $X$ est un filtre $\mathcal{F}$ sur $X$ qui vérifie la condition supplémentaire que pour tout $A \subset X$, soit $A \subset \mathcal{F}$ soit $A^c \subset \mathcal{F}$
\end{definition}

\begin{proposition}
    Un filtre est maximal pour l'inclusion si et seulement si c'est un ultrafiltre.
\end{proposition}

\begin{preuve}
    Si $\mathcal{F}$ est maximal pour l'inclusion et qu'il existe $A$ tel que ni $A$ ni $A^c$ ne soit dans $\mathcal{F}$, on a alors pour tout $F \in \mathcal{F}$ que $F \cap A \neq \varnothing$, sinon $F \subset A^c$ et donc $A^c \subset \mathcal{F}$. Ainsi $\mathcal{F} \cup \{A\}$ a la propriété de l'intersection finie et le filtre engendré est strictement plus grand que $\mathcal{F}$.\\

    Réciproquement, si $\mathcal{F}$ est un ultrafiltre, on ne peut avoir un filtre strictement plus fin que $\mathcal{F}$, puisqu'il contiendrait un ensemble $A \not\in \mathcal{F}$ et on aurait alors $A$ et $A^c$ contenu dans un seul filtre, donc $A \cap A^c = \varnothing$, absurde.\\
\end{preuve}

On connaît déjà des ultrafiltres, à savoir les filtres principaux sur les singletons. Les autres exemples qu'on n'a vu ne sont pas des ultrafiltres. Peut-on en trouver d'autres, sur un ensemble infini ? Oui, mais pas sans le choix. On peut en fait montrer qu'il existe des modèles de $\mathbf{ZF}$ sans ultrafiltre non principal sur $\mathbb{N}$. Ici, nous utiliserons sans y penser à deux fois l'axiome du choix dans toute sa généralité:

\begin{proposition}
    Tout filtre $\mathcal{F}$ est contenu dans un ultrafiltre
\end{proposition}

\begin{preuve}
    Une union croissante de filtres étant encore un filtre (exercice !). Zorn sur l'ensemble des filtres plus fins que $\mathcal{F}$ fournit alors le résultat.\\
\end{preuve}

Pour $X$ un ensemble muni d'un (ultra)filtre $\mathcal{F}$, et $f: X \to Y$, on peut "transporter" l'(ultra)filtre sur $X$ à $Y$:
\begin{proposition}
    Si $\mathcal{F}$ est un (ultra)filtre sur un ensemble $\mathcal{X}$ et $f: X \to Y$ une application, $\mathcal{F}_f = \{A \subset Y \mid f^{-1}(A) \in \mathcal{F}\}$ est un (ultra)filtre sur $Y$.
\end{proposition}

\begin{preuve}
    La condition $\varnothing \not\in \mathcal{F}_f$ est claire, et $f^{-1}(Y) = X \in \mathcal{F}$ donne $Y \in \mathcal{F}_f$. Si $A, B \in \mathcal{F}_f$, alors $f^{-1}(A \cap B) = f^{-1}(A) \cap f^{-1}(B)$ donne $A \cap B \in \mathcal{F}_f$. L'image réciproque est croissante selon l'inclusion, donc $\mathcal{F}_f$ est clos par sur-ensemble. Finalement, si $\mathcal{F}$ est un ultrafiltre et que $f^{-1}(A) \not \in \mathcal{F}$, alors $f^{-1}(A^c) = f^{-1}(A)^c \in \mathcal{F}$ et donc $A^c \in \mathcal{F}_f$.
\end{preuve}

\section{Les filtres en topologie}
Maintenant que les définitions de base sont posées, on s'intéresse à l'utilisation des (ultra)filtres en topologie, avec comme ligne de mire la preuve du fameux théorème de Tychonov, qui énonce qu'un produit (quelconque) de quasi-compact est quasi-compact pour la topologie produit.
\begin{proposition}
    Soit $X$ un espace topologique. L'ensemble des voisinages $\mathcal{V}(x)$ d'un point $x \in X$ est un filtre
\end{proposition}
\begin{preuve}
    $\varnothing$ ne contient pas $x$ donc n'est pas un voisinage de $x$. $X$ est un ouvert contenant $x$, donc un voisinage de $x$. L'intersection de deux voisinages de $x$ est encore un voisinage de $x$, puisque l'intersection de deux ouverts contenant $x$ est encore un ouvert contenant $x$. Finalement un sur-ensemble d'un voisinage de $x$ est clairement encore un voisinage de $x$.\\
\end{preuve}

On peut alors définir une notion de convergence d'un filtre:
\begin{definition}
    On dit qu'un filtre $\mathcal{F}$ sur un espace topologique $X$ converge vers un point $x \in X$ si il est plus fin que son filtre des voisinages.
\end{definition}

On note que cette définition généralise la notion de convergence d'une suite dans un espace topologique. En effet, si $f: \bN \to X$ est une suite et qu'on muni $\bN$ du filtre cofini qu'on note $\mathcal{F}^{\text{cofinite}}$, alors le filtre $(\mathcal{F}_f^{\text{cofinite}})$ converge vers $x$ si et seulement si $f$ converge vers $x$. Vérifiez le !\\

\begin{proposition}
    Un filtre $\mathcal{F}$ sur un espace topologique $X$ converge vers $x \in X$ si et seulement si pour tout ouvert $U$ contenant $x$, $U \in \mathcal{F}$.
\end{proposition}

\begin{preuve}
    Le filtre engendré par les ouverts contenant $x$ est $\mathcal{V}(x)$.\\
\end{preuve}

On a le (très beau) théorème suivant, qui caractérise la compacité et la séparabilité d'un espace en fonction de la convergence des ultrafiltres sur cet espace.\\
On note immédiatement que la première équivalence rappelle le cas métrique, avec le théorème de Bolzano-Weierstrass, qui énonce que toute suite dans un métrique compact admet une valeur d'adhérence.
Le sens réciproque de la seconde équivalence rappelle quant à lui l'unicité de la limite d'une suite dans un espace séparé.\\
\begin{theoreme}
    Soit $X$ un espace topologique.
    \begin{enumerate}
        \item $X$ est quasi-compact si et seulement si tout ultrafiltre sur $X$ converge vers au moins un point.
        \item $X$ est séparé si et seulement si tout ultrafiltre sur $X$ converge vers au plus un point.
    \end{enumerate}
\end{theoreme}
\begin{preuve}
    Soit $X$ un espace quasi-compact et $\mathcal{F}$ un ultrafiltre ne convergeant vers aucun point $x \in X$. On a pour tout $x \in X$ un ouvert $U_x$ contenant $x$ tel que $U_x \not\in \mathcal{F}$, et $(U_x)_{x \in X}$ recouvre $X$. On en extrait un sous-recouvrement fini $U_{x_1}, \ldots, U_{x_n}$.\\
    Puisqu'on a $U_{x_i}^c \in \mathcal{F}$ pour tout $i$, on a également $\bigcap_{i=1}^{n}{U_{x_i}^c} = \left(\bigcup_{i=1}^{n}{U_{x_i}}\right)^c = X^c = \varnothing \in \mathcal{F}$, absurde.\\

    Réciproquement, si on a $X$ un espace vérifiant que tout ultrafiltre converge vers un point, et qu'on a $(U_i)$ un recouvrement de $X$ n'admettant aucun sous-recouvrement fini, on peut considérer la famille $(U_i^c)$, qui a la propriété de l'intersection finie. Ainsi, il existe un ultrafiltre la contenant, et qui ne peut converger vers aucun point $x \in X$, puisqu'il existe $i$ tel que $x \in U_i$ et $U_i$ ne peut être dans cet ultrafiltre.\\

    Soit $X$ un espace séparé et $\mathcal{F}$ un ultrafiltre convergent vers un point $x \in X$.
    Soit $y \in X$, il existe $U_x, U_y$ tels que $x \in U_x, y \in U_y$ et $U_x \cap U_y = \varnothing$, par définition de la séparabilité. Ainsi $U_y \not\in \mathcal{F}$ et donc $\mathcal{F}$ ne converge pas vers $y$.\\

    Réciproquement, soit $X$ un espace non séparé dans lequel tout ultrafiltre converge vers au plus un point.
    On trouve $x, y \in X$ tels que pour tout ouvert $U_x$ contenant $x$ et tout ouvert $U_y$ contenant $y$, $U_x \cap U_y \neq \varnothing$. On a alors que l'ensemble des ouverts contenant $x$ ou $y$ a la propriété de l'intersection finie, donc qu'ils engendrent un filtre, à savoir $\mathcal{V}(x) \cup \mathcal{V}(y)$, puis que ce filtre est contenu dans un ultrafiltre, qui converge à la fois vers $x$ et $y$. Absurde.
\end{preuve}

Muni de ce théorème, on peut désormais montrer le théorème de Tychonov presque sans effort !\\
Un petit lemme rapide avant ça, qu'on pourrait prouver par les filtres en utilisant le théorème plus haut mais qu'on va traiter à la main
\begin{lemme}
    Si $(X_i)_{i \in I}$ est une famille d'espaces topologiques séparé, alors le produit $X = \prod_{i \in I}{X_i}$ est lui-même séparé.
\end{lemme}
\begin{preuve}
    Si $(x_i)_{i \in I}, (y_i)_{i \in I}$ sont deux éléments distincts de $X$, alors il existe $j \in I$ tels que $x_j \neq y_j$. Par séparation de $X_j$, on trouve donc $U_j^x, U_j^y$ ouverts dans $X_j$ tels que $x \in U_j^x, y \in U_j^y$ et $U_j^x \cap U_j^y = \varnothing$.\\
    On considère alors $U^x = U_j^x \times \prod_{i \in I\setminus\{j\}}{X_i}$ et $U^y = U_j^y \times \prod_{i \in I\setminus\{j\}}{X_i}$, qui sont tous deux ouverts dans $X$, disjoints, et contenant respectivement $(x_i)_{i \in I}$ et $(y_i)_{i \in I}$.
\end{preuve}

\begin{theoreme}[Tychonov]
    Soit $(X_i)_{i \in I}$ une famille d'espaces topologiques (quasi-)compacts. Alors $X = \prod_{i \in I}{X_i}$ est (quasi-)compact pour la topologie produit.
\end{theoreme}
\begin{preuve}
    Soit $\mathcal{F}$ un ultrafiltre sur $X$ et $\pi_i: X \to X_i$ les projections.
    On a, pour chaque $i \in I$, $\mathcal{F}_{\pi_i}$ qui est un ultrafiltre sur $X_i$, qui admet alors une limite $x_i$. \\
    On veut alors montrer que $(x_i)_{i \in I}$ est une limite de $\mathcal{F}$, donc montrer que tout ouvert de $X$ contenant $(x_i)$ est contenu dans $\mathcal{F}$.\\
    Les ouverts de la topologie produit étant formé des unions d'ouverts élémentaires, à savoir des ouverts de la forme
    $$U_{j_1}\times \ldots \times U_{j_n} \times \left(\prod_{i \in I\setminus\{j_1, \ldots, j_n\}}{X_i}\right) \text{ où } U_{i_1}, \cdots, U_{i_n} \text{ ouvert dans respectivement } X_{i_1}, \ldots, X_{i_n}$$
    et $\mathcal{F}$ étant clos par sur-ensemble, montrer que les ouverts élémentaires contenant $(x_i)_{i \in I}$ appartiennent à $\mathcal{F}$ suffit. Mieux encore, puisque tous les ouverts élémentaires sont intersection finie d'ouverts de la forme $U_j \times \prod_{i \in I\setminus\{j\}}{X_i}$ avec $U_j$ ouvert de $X_j$ et que $\mathcal{F}$ est stable par intersection finie,  il suffit de seulement montrer que ces ouverts sont dans $\mathcal{F}$.\\
    Soit $U_j \in X_j$ un ouvert contenant $x_j$. Puisque $U_j \in \mathcal{F}_{\pi_j}$, $\pi_j^{-1}(U_j) = U_j \times \prod_{i \in I\setminus\{j\}}{X_i} \in \mathcal{F}$, ce qui conclut le cas quasi-compact.\\
    Le cas compact suit alors du lemme précédent.
\end{preuve}
\end{document}
